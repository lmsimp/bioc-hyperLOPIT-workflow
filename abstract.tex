Spatial proteomics is the systematic study of protein sub-cellular
localisation. In this workflow, we describe the analysis of a typical
quantitative mass spectrometry-based spatial proteomics experiment
using the \Biocpkg{MSnbase} and \Biocpkg{pRoloc} Bioconductor package
suite. To walk the user through the computational pipeline, we use a
recently published experiment predicting protein sub-cellular
localisation in pluripotent embryonic mouse stem cells. We describe
the software infrastructure at hand, importing and processing data,
quality control, sub-cellular marker definition and visualisation. We
then demonstrate the application and interpretation of statistical
learning methods, including novelty detection using semi-supervised
learning, classification, clustering and transfer learning and
conclude the pipeline with data export. The workflow is aimed at
beginners who are familiar with proteomics in general and spatial
proteomics in particular.
